\documentclass{article}

\usepackage{amsmath}
\usepackage{graphicx}
\usepackage{hyperref}
\usepackage{csvsimple}
\usepackage{longtable}
\usepackage{booktabs}

\begin{document}

\title{Phân Tích CSV}
\author{Một Mô Hình Ngôn Ngữ Lớn}
\date{\today}

\maketitle

\section{Giới thiệu}

Tài liệu này trình bày cách phân tích dữ liệu từ tệp CSV bằng \LaTeX. Chúng ta sẽ sử dụng gói \texttt{csvsimple} để đọc và hiển thị dữ liệu.

\section{Ví dụ về Tệp CSV}

Giả sử chúng ta có một tệp CSV có tên là \texttt{data.csv} với nội dung sau:

\begin{verbatim}
Name,Age,City
Alice,30,New York
Bob,25,London
Charlie,35,Paris
\end{verbatim}

\section{Sử dụng gói \texttt{csvsimple}}

Chúng ta có thể sử dụng gói \texttt{csvsimple} để đọc và hiển thị dữ liệu này trong một bảng.

\subsection{Bảng cơ bản}

\begin{center}
\csvreader[head to column names]{data.csv}{}{%
  \Name & \Age & \City \\\hline
}
\end{center}

\subsection{Bảng với \texttt{longtable} và \texttt{booktabs}}

Để tạo ra một bảng dài hơn và đẹp mắt hơn, chúng ta có thể sử dụng gói \texttt{longtable} và \texttt{booktabs}.

\begin{longtable}{lll}
\caption{Dữ liệu từ \texttt{data.csv}} \\
\toprule
Name & Age & City \\
\midrule
\endfirsthead

\multicolumn{3}{c}%
{{\bfseries Tiếp tục từ trang trước}} \\
\toprule
Name & Age & City \\
\midrule
\endhead

\midrule
\multicolumn{3}{r}{{Tiếp tục ở trang sau}} \\
\endfoot

\bottomrule
\endlastfoot

\csvreader[head to column names]{data.csv}{}{%
  \Name & \Age & \City \\
}
\end{longtable}

\section{Kết luận}

\LaTeX, cùng với các gói như \texttt{csvsimple}, có thể được sử dụng để phân tích và trình bày dữ liệu từ tệp CSV một cách hiệu quả.

\end{document}
