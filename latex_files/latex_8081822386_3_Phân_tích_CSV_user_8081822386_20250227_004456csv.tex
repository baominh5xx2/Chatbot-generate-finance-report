\documentclass[12pt]{article}
\usepackage{amsmath}
\usepackage{graphicx}
\usepackage{hyperref}
\usepackage{geometry}
\usepackage{booktabs}
\usepackage{float}

\geometry{a4paper, margin=1in}

\title{Báo Cáo Phân Tích Dữ Liệu CSV: user\_8081822386\_20250227\_004456.csv}
\author{}
\date{\today}

\begin{document}

\maketitle
\thispagestyle{empty}

\newpage
\tableofcontents
\newpage

\section{Tổng quan về dữ liệu}

\begin{itemize}
    \item Tên file: user\_8081822386\_20250227\_004456.csv
    \item Số hàng: 100
    \item Số cột: 2
    \item Các cột: X, Y
\end{itemize}

\section{Phân tích chi tiết từng cột quan trọng}

\subsection{Cột X}

\begin{itemize}
    \item Min: 1.00
    \item Max: 100.00
    \item Mean: 50.50
    \item Std: 29.01
\end{itemize}

\subsection{Cột Y}

\begin{itemize}
    \item Min: 1.00
    \item Max: 92.00
    \item Mean: 47.85
    \item Std: 27.02
\end{itemize}

\section{Phát hiện các mẫu và xu hướng trong dữ liệu}

Dựa trên các thống kê mô tả, cột X và Y có phân bố khá rộng.  Để hiểu rõ hơn về mối quan hệ giữa X và Y, cần thực hiện thêm các phân tích như biểu đồ phân tán (scatter plot) và tính hệ số tương quan.  Có thể có một xu hướng tuyến tính hoặc phi tuyến tính giữa hai cột.  Việc kiểm tra các giá trị ngoại lệ (outliers) cũng rất quan trọng.

\section{Đề xuất các phương pháp phân tích sâu hơn}

\begin{itemize}
    \item \textbf{Phân tích tương quan:} Tính hệ số tương quan Pearson hoặc Spearman để xác định mức độ liên kết tuyến tính hoặc đơn điệu giữa X và Y.
    \item \textbf{Biểu đồ phân tán (Scatter Plot):}  Vẽ biểu đồ phân tán giữa X và Y để trực quan hóa mối quan hệ giữa chúng.
    \item \textbf{Hồi quy (Regression):} Nếu có một mối quan hệ rõ ràng giữa X và Y, có thể xây dựng mô hình hồi quy để dự đoán giá trị của Y dựa trên X.
    \item \textbf{Phân tích cụm (Clustering):}  Nếu dữ liệu có nhiều chiều hơn (sau khi bổ sung thêm cột), phân tích cụm có thể giúp xác định các nhóm dữ liệu có đặc điểm tương đồng.
    \item \textbf{Phân tích chuỗi thời gian (Time Series Analysis):} Nếu dữ liệu được thu thập theo thời gian, phân tích chuỗi thời gian có thể giúp phát hiện các xu hướng và chu kỳ.
\end{itemize}

\section{Kết luận}

Báo cáo này cung cấp một tổng quan ban đầu về dữ liệu CSV "user\_8081822386\_20250227\_004456.csv". Các thống kê mô tả cho thấy sự phân bố của các cột X và Y. Để hiểu sâu hơn về dữ liệu, cần thực hiện thêm các phân tích như phân tích tương quan, hồi quy và trực quan hóa dữ liệu.  Việc xác định và xử lý các giá trị ngoại lệ cũng rất quan trọng để đảm bảo tính chính xác của các phân tích.
\end{document}
